\documentclass{article}

\title{COMP26120 Lab 10}
\author{Daronhil Mauricette}

\begin{document}
\maketitle

% PART 1 %%%%%%%%%%%%%%%%%%%%%%%%%%%%%%%%%%%%%%%%%%%%%%%%%%%%%%%%%%%%%%%%%%%%%%

\section{The small-world hypothesis}
\label{sec:small world}
% Here give your statement of the small-world hypothesis and how you
% are going to test it.
The small-world hypothesis is based on the theory that one can find the shortest
path from one node to any other node in that graph. We can test this using
Dijkstra's algorithm to check whether the between two nodes is the shortest path
between. 


\section{Complexity Arguments}
\label{sec:complexity}
% Write down the complexity of Dijkstra's algorithm and of Floyd's algorithm.
% Explain why, for these graphs, Dijkstra's algorithm is more efficient.
The complexity of Dijkstra's alorithm using a doubly linked list as the priority
is O(n^2). This makes it more efficient than the Floyd-Warshall algorithm that
has a complexity of O(n^3).
% PART 2 %%%%%%%%%%%%%%%%%%%%%%%%%%%%%%%%%%%%%%%%%%%%%%%%%%%%%%%%%%%%%%%%%%%%%%

\section{Part 2 results}
\label{sec:part2}
% Give the results of part two experiments.


% PART 3 %%%%%%%%%%%%%%%%%%%%%%%%%%%%%%%%%%%%%%%%%%%%%%%%%%%%%%%%%%%%%%%%%%%%%%

\section{Part 3 results}
\label{sec:part3}
% Give the results of part three experiments.


\section{Conclusions}
\label{sec:conclusions}
% Give your conclusions from the above experiments 


\end{document}
